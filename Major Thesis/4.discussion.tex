The main aim of the work is to determine which tool is more suitable to enhance the biological insight from lab-derived data in order to both serve as starting point for the research and as feedback for the wet lab.

\subsection{Framework}
The established framework allows quick information analysis. By having data available at glance, decisions are made more fluently, since the desired insight we want to focus on is build straight away.

\subsubsection{Implementation and accuracy}
Considering the IUT component assessment low score we can draw some conclusions. Despite the many checks performed, the gene symbols have been translated to internal IDs and re-translated back to the original using open source libraries. The efficiency of the procedure might have been compromised due to the large amount of items (because we are unable to review them all), but also due to the fact an internal ID, when translated back to the symbols scheme, delivered either N results or an alias symbol. E.g. we have encountered JUN proto-oncogene refered as AP1 sometimes. Corrections at this point have been made to stick to the official identifiers, yet loss of information can be possible, and in turn responsible for low scoring at the IUT Benchmark Core.
\\

As for DES-nA component, results often deliver a central node highly connected to the remaining set, hence we either assume the outputs cannot be reliable based on the fact that the mere nature of pathways is not to have an overall central influence, but rather an interconnected matrix, or that there’s a hidden variable which biases the graph to expand to an extended pathways.
\\

K  parameter for DEGAS sets the number of dysregulated genes that the algorithm will accept. This point can be the main bias of the procedure, since it limits the ability of building a larger more rich — and  probably entangled— graph around. Again, K is the most time-consuming parameter when estimating, which is mainly controlled by he number of iterations we use to predict it. Due to that, both can be considered as accuracy restrictors.
\\

Also, DIAMOnD work claim to might be building new disease associations whose morphology and content can be completely new. Lacking a ground truth to resolve its reliability and keeping in mind system’s complexity, any scenario can be plausible. Translated to our project could mean the \textit{de novo} derived networks can be coming up with a new unknown solution or just be completely inaccurate. 
\\

As a final word, and based on lack of consensus among algorithms, in the DES-nA component we can assume at least one of them is completely inaccurate. E.g. given A as an accurate result of the component and I as an inaccurate one: (A1, A2) tuple will highly overlap, (A1, I2) and (I1, A2) will differ largely, and (I1, I2) can tottaly overlap or diverge. Therefore, at least one suffers from accuracy absence.

\subsection{Possible future directions}
SigNet and MARINa can provide proper start points for DIAMOnD. Also, SigNet has a graph expansion module, which can be overlapped with DEGAS to ensure integrity.
