\subsection{Extra Insight Results}

\subsection{CEL Files treatment}
CMAP2 [reference] relates gene expression to drug targets. It disposes of raw-data “.CEL” formatted files generated by scanning a single Affymetrix microarray.

An instance table is also available to provide the relation between CEL files and lab-specific data.

CEL files were downloaded from CMAP2’ webpage via R scripting. The Instances table is also available in the same site and serves as coordinator of the case-control comparison, as well as for correctly labelling the cases.

IDMAPGen is an R package function developed at PMI. The procedure of the internal calculations held within is out of the scope of this project. It requires the CEL formatted objects as input and a coordination file, as the Instances table.

The label name is composed by collapsing together the cmap\_name, molar concentration, and cell type, and is used to tag the classes. For the CEL files naming, note the perturbation column includes a large identification number and a second label after a full stop. These are the names of the CEL files. The controls corresponding to each perturbation are formed by the same identification number and each label available in the control column. E.g. perturbation = 123.H1, control = .A1.B2.C3, then the CEL files for the case is 123.H1 and for its controls are 123.A1, 123.B2 and 123.C3. 

Due to package unavailability and compatibility issues with our current version of R Studio, only HT\_HG-U133A array derived cases have been considered which accounts for the 65\% of the total CEL files available in CMAP. The procedure delivers 1991 test cases to use as benchmark which we consider sufficient.

The calculation of the final objects results in a list of cases, each containing at least the fold-change of each analyzed gene and its p-value. The naming of each gene has been supported by external BioConductor library frmavecs, Affymetrix reader functions and biomaRt homo\_sapiens\_ensembl dataset.

\subsection{Network Exploratory Analysis}
