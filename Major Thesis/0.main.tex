\documentclass{jot}

\usepackage[utf8]{inputenc}
\usepackage[T1]{fontenc}
\usepackage[english]{babel}
\usepackage{subfiles}
\usepackage{titlesec}
\setcounter{secnumdepth}{4}
\titleformat{\paragraph}
{\normalfont\normalsize\bfseries}{\theparagraph}{1em}{}
\titlespacing*{\paragraph}
{0pt}{3.25ex plus 1ex minus .2ex}{1.5ex plus .2ex}

\usepackage{microtype} % optional, for aesthetics
\usepackage{tabularx} % nice to have
\usepackage{booktabs} % necessary for style
% \usepackage{graphicx}
% \graphicspath{{./figures/}}
% \usepackage{listings}
% \lstset{...}

% \newcommand\code[1]{\texttt{#1}}
% \let\file\code


%%% Article metadata
\title{Implementation, benchmarking and application of functional analysis workflows of omics data}

\runningtitle{Implementation, benchmarking and application of functional analysis workflows of omics data}

\author[affiliation=orgname, nowrap] % , photo=FILE]
    {Raul I. Coroban}
    {is ...
    Contact him at \email{r.coroban@student.vu.nl}}

%\affiliation{orgname}{ORGANISATION}

\runningauthor{R.I. Coroban}

%\jotdetails{
%    volume=V,
%    number=N,
%    articleno=M,
%    year=2011,
%    doisuffix=jot.201Y.VV.N.aN,
%    license=ccbynd % choose from ccby, ccbynd,ccbyncnd
%}


\begin{document}

\begin{abstract}
This work examines lung omics analyses from a 6-month inhalation exposure study with ApoE-/- mice  and enrich them with known curated data to increase biological insight. We describe the developed pipelines that implement state-of-the-art algorithms which rely on transcriptomics data to predict upstream targets and/or derive custom pathways. All procedures produce graph visualisations that we use for algorithm selection and result interpretation. Transcriptomics data has been obtained from a PMI study \cite{Titz2020Multi-omicsSmoke} while benchmark data is retrieved from online sources \cite{Wang2020TherapeuticTherapeutics} \cite{Subramanian2017AProfiles}. After benchmarking and in-depth analysis, algorithms which predict upstream regulators based on transcriptomics outperform the de novo pathway building ones, both on time efficiency and result insight.
\end{abstract}

\keywords{omics; data analysis.}

\section{Background}
\label{section:background}
\subfile{1.background.tex}

\section{Methods}
\label{section:methods}
\subfile{2.methods.tex}

\section{Results}
\label{section:results}
\subfile{3.results.tex}

\section{Discussion}
\label{section:discussion}
\subfile{4.discussion.tex}

\section{Conclusions}
\label{section:conclusions}
\subfile{5.conclusions.tex}

\section{Acknowledgements}
\subfile{6.acknowledgements.tex}

\section{Availability}
\subfile{7.availability.tex}

\bibliographystyle{abbrv}
\bibliography{references.bib}

%\abouttheauthors

%\begin{acknowledgments}
%ACKS
%\end{acknowledgments}

\end{document}
