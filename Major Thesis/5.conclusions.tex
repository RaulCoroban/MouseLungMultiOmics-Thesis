Based on the results and later analysis, we can conclude the analysed algorithms increase the overview biological insight of the transcriptomics data, and make contrasts comparable. This allows us to build the next steps for research’s future directions. Besides, SigNet and MARINa agree on top results, which adds stability to the result set.

However, at the reliability level, both assessed algorithms SigNet and MARINa deliver an insufficient score. After a careful review of the original work, we can say this can be either due to errors regarding gold standard construction or simply due post-benchmark calculations, triggered by normalisations and generalisation misconceptions.

DEGAS and DIAMOnD results often deliver a central node we either assume the outputs cannot be reliable, or that there’s hidden knowledge we are overseeing and which biases the graph development. Since any of these have been assessed, there’s no reliability score to discuss.

Overall, we can agree that network analysis increases specific insight about a focus point.
